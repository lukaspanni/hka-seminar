\section{Fazit}
\label{sec:Fazit}

Apples „Wo ist?“ Dienst nutzt den großen Marktanteil aus, um Crowdsourced-Tracking verlorener Apple-Gerät zu ermöglichen.
Im Vergleich zu bisher verfügbaren Diensten erhöht „Wo ist?“ den Schutz der Privatsphäre seiner Nutzer.
Insbesondere werden die Standorte verlorener Geräte mit einer Ende-zu-Ende-Verschlüsselung vor unbefugtem Zugriff geschützt.
Weitere Maßnahmen werden ergriffen, um die Ausnutzung des Dienstes für das Tracking von Personen durch Dritte zu verhindern.

Allerdings konnte gezeigt werden, dass die von Apple getroffenen Maßnahmen nicht jeden Missbrauch verhindern können.
Vorhandene Sicherheitsmaßnahmen zur Verhinderung von direktem Tracking durch Dritte beziehen sich nur auf iOS-Geräte und können demnach die Mehrheit der Personen nicht schützen.
Weiterhin können diese Sicherheitsmaßnahmen mit geringem technischem Aufwand umgangen werden.
Die grundlegende Funktionsweise des „Wo ist?“ Dienstes macht es unmöglich das Umgehen der Sicherheitsmaßnahmen zu verhindern.
Nur eine erhebliche Anpassung des gesamten Dienstes hat das Potenzial diesen Missbrauch zu verhindern.

Neben direktem Tracking lässt sich der Dienst auch für ein indirektes Tracking der Nutzer einsetzen.
Dieses Missbrauchsszenario lässt sich allerdings auch für positive Zwecke, wie Crowd-Monitoring unter Erhalt der Privatsphäre, nutzen.
Den negativen Missbrauch könnte Apple hier mit geringem Aufwand verhindern.

Nutzer können kaum Maßnahmen ergreifen, um sich selbst vor den Auswirkungen des Missbrauchs zu schützen.
Das Opt-Out aus dem Dienst oder die Deaktivierung der Bluetooth-Funktionalität schützt vor allen Missbrauchsszenario mit Ausnahme des direkten Trackings durch Dritte.
Stattdessen wird das direkte Tracking durch diese Maßnahme erleichtert.
Demnach kann es keine klare Empfehlung für die Nutzer geben, wie sie sich vor Auswirkungen von Missbrauch des Dienstes schützen können.


Obwohl verschiedene Szenarien für den Missbrauch des Dienstes aufgezeigt wurden, stellt der „Wo ist?“ Dienst eine Verbesserung gegenüber bisherigen Diensten dar.
Die Standortdaten verlorener Geräte sind weder Apple noch Angreifern zugänglich.
In der Vergangenheit konnte gezeigt werden, dass dies für viele Konkurrenzdienste nicht gilt \cite{Garg_Secure_Tracker,Weller_BLE_Finders}.
Dennoch ist der Dienst durch die große Verbreitung von Apple-Geräten auch für Angreifer attraktiv.
Deshalb ist es notwendig, dass Apple die vorhandenen Sicherheitsmaßnahmen ausbaut und weitere Maßnahmen ergreift.
Verschiedene Maßnahmen wurden bereits vorgeschlagen, die zumindest dafür geeignet sind, den Missbrauch zu erschweren.
Weiterhin gibt es verschiedene Vorschläge zu sicheren Crowdsourced-Tracking-Diensten, welche nicht für den hier vorgestellten Missbrauch anfällig sind \cite{Garg_Secure_Tracker,Weller_BLE_Finders}.
Diese Ressourcen sollte Apple für die Weiterentwicklung des „Wo ist?“ Dienstes in Betracht ziehen.