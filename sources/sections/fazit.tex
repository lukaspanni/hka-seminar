\section{Fazit}
\label{sec:Fazit}

Mit dem „Wo ist?“ Dienst bietet Apple eine Lösung zum Crowdsourced-Tracking verlorener Geräte an, die aufgrund der großen Verbreitung von Apples Produkten hohe Standortgenauigkeit bietet.
Im Vergleich zu vor der Einführung von „Wo ist?“ existierenden Konkurrenzdiensten schützt der Dienst die Privatsphäre seiner Nutzer besser.
Insbesondere werden Standortinformationen immer durch eine Ende-zu-Ende-Verschlüsselung von unbefugtem Zugriff geschützt.
Darüber hinaus werden weitere Maßnahmen getroffen, die den Missbrauch des Dienstes, zum Beispiel für unberechtigtes Tracking von Personen, zu verhindern.
Zwar konnten bisher keine Schwachstellen in der eingesetzten Verschlüsselung identifiziert werden, jedoch konnte bereits mehrfach gezeigt werden \cite{Mayberry_Tracking,braeunlein_sendmy,Tonetto_FindMy}, dass die vorhandenen Maßnahmen zur Verhinderung von Missbrauch nicht ausreichen.

So sind beispielsweise die bestehenden Maßnahmen gegen direktes Tracking, wie in \autoref{missbrauch:3} gezeigt, nur eingeschränkt für Nutzer außerhalb des Apple-Ökosystems offiziell verfügbar und lassen sich darüber hinaus mit geringem Aufwand unentdeckt umgehen.
Damit eignet sich der „Wo ist?“ Dienst sehr gut für Stalking oder ähnliche Zwecke.
Gegen dieses Problem können sich Nutzer nicht wirksam schützen.
Apple kann nur durch eine erhebliche Anpassung des Designs des aktuellen Dienstes diesen und weiteren Missbrauch sicher verhindern.

Weitere Missbrauchsszenarien lassen sich leichter verhindern.
Negatives indirektes Tracking wie in \autoref{missbrauch:4} vorgestellt, könnte beispielsweise durch ein einfaches Software-Update der Server des Dienstes behoben werden.


Gegen keines der in \autoref{sec:Missbrauch} aufgeführten Missbrauchsszenarien, können sich Nutzer mit geringem Aufwand und ohne Verzicht auf die Verwendung des Dienstes schützen.
Selbst das Opt-out aus dem Dienst oder die Deaktivierung der Bluetooth-Funktionalität schützt den Nutzer nicht vor allen Missbrauchsszenarien.
Stattdessen wird so das direkte Tracking schwerer erkennbar.
Demnach kann es keine klare Empfehlung für die Nutzer geben, wie sie sich vor Auswirkungen von Missbrauch des Dienstes schützen können.
Dennoch stellt der Dienst im Vergleich zu Konkurrenzdiensten vor der Einführung von „Wo ist?“ eine deutliche Verbesserung dar.
Die Standortdaten verlorener Geräte sind weder Apple noch Angreifern zugänglich.
In der Vergangenheit konnte gezeigt werden, dass dies für viele Konkurrenzdienste nicht gilt \cite{Garg_Secure_Tracker,Weller_BLE_Finders}.
Trotzdem muss Apple aktiv werden und zumindest gegen die Missbrauchsszenarien, die sich mit geringem Aufwand verhindern lassen, geeignete Maßnahmen treffen.
Wie bereits mehrfach erwähnt lassen sich nicht alle Missbrauchsszenarien einfach verhindern.
Gegen diese existieren jedoch Maßnahmen, welche das Ausnutzen der Schwachstellen erschweren, beziehungsweise die Auswirkungen des Missbrauchs verringern können.
Auch diese Maßnahmen sollte Apple umsetzen.

Weiterhin gibt es verschiedene Vorschläge zu sicheren Crowdsourced-Tracking-Diensten, welche nicht für den hier vorgestellten Missbrauch anfällig sind \cite{Garg_Secure_Tracker,Weller_BLE_Finders}.
Durch die große Verbreitung des standardmäßig aktivierten „Wo ist?“ Dienstes und Apples restriktiven \acp{API} haben Konkurrenzdienste, welche diese Maßnahmen umsetzen, jedoch kaum eine Chance, eine nennenswerte Verbreitung zu erreichen.
Deshalb sollte Apple zur Weiterentwicklung des Dienstes auf die Vorschläge zu sicheren Diensten zurückgreifen und den Dienst sicherer zu machen, um dem Marketing zu entsprechen.