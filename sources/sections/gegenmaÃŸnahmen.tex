\section{Gegenmaßnahmen}
\label{sec:Gegenmassnahmen}

Im Folgenden werden mögliche Gegenmaßnahmen gegen die Szenarien \nameref{missbrauch:1} bis \nameref{missbrauch:6} vorgestellt.
Dabei wird jeweils angegeben, ob eine Gegenmaßnahme durch die Nutzer möglich ist oder, ob Apple Maßnahmen ergreifen muss.
In diesem Zusammenhang wird auch auf Schwachstellen der einzelnen Gegenmaßnahmen eingegangen.

Die Nutzer des „Wo ist?“ Dienstes können sich gegen alle Missbrauchsszenarien außer \nameref{missbrauch:3} schützen, indem sie den Dienst nicht nutzen und der Teilnahme als Finder Devices widersprechen.
Diese Gegenmaßnahme wird jedoch nicht als sinnvoll angesehen, da der Dienst potenziell sehr nützlich sein kann.
Stattdessen werden, wenn möglich, Gegenmaßnahmen vorgestellt, die eine möglichst uneingeschränkte Nutzung des Dienstes ermöglichen.
Es konnten keine Gegenmaßnahmen identifiziert werden, welche gegen alle Missbrauchsszenarien gleichzeitig wirksam sind.

Dennoch gibt es eine Gegenmaßnahme, welche \nameref{missbrauch:5} sicher verhindert und das Umgehen der bestehenden Maßnahmen gegen \nameref{missbrauch:3} verhindert.
Dazu muss der Einsatz inoffizieller Tracker verhindert werden.
Zur sicheren Verhinderung des Einsatzes von inoffiziellen Trackern müsste Apple beispielsweise eine Authentifizierung für alle Tracker umsetzen, was laut Mayberry \textit{et al.} \cite{Mayberry_Tracking} nicht ohne erhebliche Anpassungen der offiziellen Tracker möglich ist.
Eine weitere Möglichkeit wäre die Registrierung aller offiziellen Tracker mitsamt \ac{MBK}, um erkennen zu können, ob ein Location Report mit einem gültigen Schlüssel erzeugt wurde.
Da diese Methode jedoch die Ende-zu-Ende-Verschlüsselung der Location Reports aufweicht, sodass Apple die Daten entschlüsseln kann, wird sie als nicht geeignet angesehen.


\subsection{Gegenmaßnahmen zu \nameref{missbrauch:1}}
Das Design des Dienstes lässt es prinzipiell zu, Replay-Angriffe zumindest zu erkennen.
Die „Wo ist?“ App könnte beispielsweise die Plausibilität der empfangenen Standortdaten prüfen und Nutzer auf eventuelle Manipulationen hinweisen.
Eine solche Prüfung wäre unter anderem anhand der Zeitpunkte der Standortdaten möglich.
Große Abweichungen des Standorts bei Reports, welche kurz nacheinander oder gleichzeitig erstellt wurden, könnten als nicht realistisch erkannt werden.
Allerdings lassen sich so nicht alle Manipulationen erkennen, da der Replay-Angriff zum Beispiel auch in der Nähe des tatsächlichen Standorts durchgeführt werden kann, sodass die Abweichungen der Standorte klein sind.
In diesem Fall wird lediglich die Genauigkeit der Lokalisierung eingeschränkt.
Außerdem könnte das Owner Device bestimmen, welcher Advertising Key zum Zeitpunkt der Generierung des Location Reports aktiv war und so überprüfen ob der Location Report mit dem passenden Advertising Key erstellt wurde, oder ob ein älterer Schlüssel verwendet wurde.
So lassen sich mögliche Replay-Angriffe auf das Intervall der Schlüsselrotation beschränken.

Jedoch sind diese Maßnahmen nur durch Apple, durch eine Aktualisierung der „Wo ist?“ App umsetzbar. 
Auf Apples Server können Manipulationen hingegen nicht erkannt werden, da durch die Verschlüsselung kein Zugriff auf die Standortinformationen besteht.

Die Nutzer haben keine Möglichkeit, sich vor einem Replay-Angriff zu schützen.
Sobald ein Gerät Advertisement-Pakete sendet, könnten diese von einem Angreifer aufgezeichnet und für einen Replay-Angriff verwendet werden.


\subsection{Gegenmaßnahmen zu \nameref{missbrauch:2}}

Schutz gegen die Entladung der Batterie durch Angreifer in der Nähe ist nur schwierig umsetzbar.
Da bei Apples System die Verschlüsselung nicht auf dem verlorenen Gerät, sondern auf dem Finder Device erfolgt, ist dieser Angriff vermutlich aufwändiger als beim System von Garg \textit{et al.} \cite{Garg_Secure_Tracker}.
Außerdem ist das Abspielen eines Tons vergleichsweise auffällig und damit für Angreifer nicht in jeder Situation gut geeignet.

Als Gegenmaßnahme könnte Apple die Funktion Töne abzuspielen entfernen, was den Angriff nicht komplett verhindern, aber erschweren kann.
Allerdings ist diese Funktion Teil der Schutzmaßnahmen gegen direktes Tracking und die Entfernung dieser Funktion würde die Maßnahmen zur \ac{UT} abschwächen.

Die Nutzer haben kaum Möglichkeiten sich gegen dieses Szenario zu schützen. 
Das Entfernen des Lautsprechers von Accessories wäre eine Möglichkeit die Entladung durch Angreifer in der Nähe zu erschweren.
Da die Entladung durch das Abspielen von Tönen jedoch auffällig und demnach als eher unwahrscheinlich anzusehen ist, ist diese Maßnahme nur bedingt sinnvoll.

% Gegen physische Angriffe auf ein Lost Device kann das Design des „Wo ist?“ Dienstes nicht schützen.
% Die Zerstörung des Geräts könnte maximal erschwert, aber nicht verhindert werden.
% Durch eine fest verbaute Batterie könnte das Entfernen der Batterie ebenfalls nur erschwert aber nicht verhindert werden.
% Da der „Wo ist?“ Dienst nicht als Hilfsmittel gegen Diebstahl konzipiert ist und nicht als solches beworben \cite{Apple_WoIst} wird, wird Apple vermutlich auch in Zukunft keine Maßnahmen gegen physische Angriffe ergreifen.

\subsection{Gegenmaßnahmen zu \nameref{missbrauch:3}}
\label{sec:gegenmassnahmen:3}

Wie bereits in \nameref{missbrauch:3} beschrieben, schützen die aktuell von Apple umgesetzte Maßnahmen zur \acl*{UT} nur unzureichend und lassen sich mit geringem Aufwand umgehen.
Allgemein setzt sich die \ac{UT} aus verschiedenen Funktionen zusammen.

Accessories müssen zum Beispiel einen Ton abspielen, wenn sie mindestens drei Tage vom Gerät des Besitzers getrennt sind und Bewegung erkennen \cite{Apple_FindMySpec}.
Da der Lautsprecher bei AirTags beispielsweise einfach entfernt werden kann, und AirTags ohne Lautsprecher auf verschiedenen Online-Marktplätzen erhältlich sind, trägt diese Funktion kaum zum Schutz vor Stalking bei \cite{Heinrich_AirGuard}.
Zusätzlich zeigen Roth \textit{et al.} \cite{Roth_airtags}, dass die Firmware von AirTags manipuliert werden kann, um diese Funktion zu deaktivieren oder anzupassen.

Weiterhin kann iOS die Verfolgung durch ein Accessory eines anderen Nutzers erkennen und eine Warnung anzeigen.
Dazu werden \ac{BLE}-Advertisements von Accesories, welche seit mehr als zehn Minuten empfangen werden und dem Gerät mindestens 840 Meter gefolgt sind, als verdächtig markiert \cite{Heinrich_AirGuard}.
Warnungen werden allerdings, zur Vermeidung von false positives, verzögert angezeigt.
Heinrich \textit{et al.} \cite{Heinrich_AirGuard} zeigen, dass die Warnungen erst mehrere Stunden nach der Erkennung oder bei Rückkehr des Benutzers an den von iOS als Zuhause bestimmten Ort angezeigt werden.
Ein Angreifer kann die Position seines Opfers also, ohne die \ac{UT} umgehen zu müssen, mindestens mehrere Stunden oder sogar bis zu dessen Zuhause verfolgen, ohne eine Warnung auszulösen.
Besitzt das Opfer kein Apple-Gerät, ist eine unbegrenzte Verfolgung möglich.

Inzwischen bietet Apple auch eine App für Android an, um die Warnung vor Tracking auch für Android-Nutzer zu ermöglichen, welche jedoch von Heinrich \textit{et al.} \cite{Heinrich_AirGuard} mangels Features nicht als ausreichend angesehen wird.
Die Open-Source App \textit{AirGuard} für Android bietet eine ähnliche Warnfunktion, welche in Tests besser abschneidet als die Warnfunktion von iOS und die offizielle App von Apple.
Für iOS ist dieser bessere Schutz aufgrund von Einschränkungen der Bluetooth-\acp{API} für Apps von Drittanbietern nicht verfügbar \cite{Heinrich_AirGuard}.

Problematisch ist ebenfalls, dass die \ac{UT} von iOS durch den Einsatz inoffizieller Tracker einfach und zuverlässig umgangen werden kann \cite{Heinrich_AirGuard,Mayberry_Tracking}.
Einerseits können inoffizielle Tracker für Status-Feld im Advertisement-Format von Apple immer den Wert 0 senden, was den Tracker als verlorenes Endgerät ausweist.
Verlorene Endgeräte werden für das \ac{UT}-Feature nicht betrachtet und generieren deshalb keine Warnungen \cite{Heinrich_AirGuard,Mayberry_Tracking}.
Die AirGuard-App hingegen warnt unabhängig vom Status-Feld \cite{Heinrich_AirGuard}, was zeigt, dass zumindest diese Möglichkeit zur Umgehung einfach behoben werden kann.

Andererseits können Tracker den Advertising Key häufig wechseln, sodass iOS nicht erkennen kann, ob es sich um den gleichen Tracker handelt.
Damit ist keine Unterscheidung von Trackern und Endgeräten ohne Internetverbindung und keine Warnungen mehr möglich \cite{Mayberry_Tracking}.
Würde auch vor Endgeräten ohne Internetverbindung gewarnt, würden vermutlich sehr viele falsche Warnungen generiert werden.
Die einzige sichere Möglichkeit, ist die Verwendung inoffizieller Tracker zu verhindern, wie oben bereits beschrieben.


Nutzer können sich vor Tracking durch offizielle Tracker warnen lassen, indem sie entweder ein iOS-Gerät, oder die AirGuard-App verwenden.
Weiterhin können inoffizielle Tracker, welche nur das Status-Feld manipulieren, von der AirGuard-App erkannt werden und Android-Nutzer können auch in diesem Fall gewarnt werden.
Durch manuelle \ac{BLE}-Scans der Umgebung haben Nutzer von AirGuard außerdem die Möglichkeit, beliebige „Wo ist?“-Geräte in der Nähe zu erkennen.
Durch die beschriebenen Einschränkungen müssen Nutzer jedoch selbst Entscheiden, ob ein Gerät als schadhafter Tracker einzustufen ist.

\subsection{Gegenmaßnahmen zu \nameref{missbrauch:4}}

Das Missbrauchsszenario \nameref{missbrauch:4} lässt sich sowohl für den Erhalt als auch für die Beeinträchtigung der Privatsphäre nutzen.
Damit sind Gegenmaßnahmen wünschenswert, welche den positiven Nutzen erhalten und gleichzeitig die mögliche negative Nutzung verhindern.
Der negative Missbrauch durch indirektes Tracking beruht darauf, Location Reports eines Finder Devices miteinander in Verbindung zu bringen.
Dazu werden die Upload-Zeitstempel verwendet, die von Apples Servern beim Empfang gespeichert werden.
Tonetto \textit{et al.} \cite{Tonetto_FindMy} schlagen daher eine Reduktion der Genauigkeit der Zeitstempel oder deren komplette Abschaffung als Gegenmaßnahmen vor.
Alternativ kann auch der Upload der einzelnen Location Reports zufällig verteilt werden.
Konkret werden zufällige Zeitintervalle oder zufällige, seit der Erstellung zurückgelegte Distanzen vorgeschlagen.

Diese Maßnahmen sind geeignet, die Zuordnung mehrerer Location Reports zu einem Finder Device zumindest zu erschweren.
Damit steigt auch der Aufwand für die Rekonstruktion von Bewegungen und die indirekte Verfolgung von Personen.

Der positive Missbrauch zum Crowd-Monitoring wird durch diese Gegenmaßnahmen nicht komplett verhindert.
Insbesondere die Abschätzung von Personendichten ist nicht davon abhängig, mehrere Location Reports zu einem Gerät zuzuordnen.
Die Bestimmung des Crowd-Flows, nach der in \cite{Tonetto_FindMy} vorgestellten Methode, ist hingegen auf diese Zuordnung angewiesen.
Allerdings ließe sich der Crowd-Flow auch anhand der Veränderung der Personendichte über die Zeit abschätzen.
Die Genauigkeit dieser Alternativen Lösung ist mangels Untersuchung durch die Autoren nicht bekannt.
Die Bestimmung der Personendichte ist nur von der Möglichkeit Location Reports abzurufen abhängig.
Damit lassen sich kaum Gegenmaßnahmen treffen, welche nicht die Funktionsweise des Dienstes an sich beeinflussen.
Die Limitierung der zurückgegebenen Anzahl von Location Reports könnte diese Methode zwar erschweren, allerdings würde so vermutlich auch die Genauigkeit der Standortinformationen von tatsächlich verlorenen Geräten reduziert werden.

Die Nutzer können sich gegen positiven und negativen Missbrauch, nach Szenario \nameref{missbrauch:4}, nur schwer schützen.
Durch regelmäßige Verbindung mit einem WLAN-Netzwerk wäre es möglich, die Zuordnung von Location Reports zu einem Gerät einzuschränken, da jeder Upload weniger Location Reports enthält.
Jedoch ist die Verbindung mit einem WLAN-Netzwerk nicht immer möglich und auch nicht immer gewünscht.
Sinnvolle Schutzmaßnahmen müssen demnach von Apple getroffen werden.

\subsection{Gegenmaßnahmen zu \nameref{missbrauch:5}}
Der verdeckte Datentransfer über den „Wo ist?“ Dienst beeinträchtigt die Privatsphäre der Nutzer nicht.
Außerdem ist nur eine sehr geringe Datenrate möglich, was dieses Missbrauchsszenario für Angreifer in der Praxis weniger interessant macht.
Dennoch hat Apple ein berechtigtes Interesse, den kostenlosen Datentransfer über seinen Dienst zu verhindern.
Wie oben bereits beschrieben lässt sich dieser Missbrauch verhindern, indem die Nutzung von inoffiziellen Trackern unterbunden wird.

Weiterhin schlagen die Autoren von \cite{Tonetto_FindMy} vor, die Anzahl der Location Reports, die von einem Finder Device hochgeladen werden können, zu begrenzen.
Durch andere Finder Devices in der Nähe, reicht diese Begrenzung jedoch nicht aus um den verdeckten Datentransfer zu verhindern.
Zusätzlich könnte die Anzahl der abrufbaren Location Reports pro Advertising Key limitiert werden.
Die Kodierung, wie von Bräunlein \cite{braeunlein_sendmy} vorgeschlagen, wäre davon allerdings nicht betroffen, da für jedes Bit der Nachricht ein eigener Advertising Key verwendet wird.

Apple könnte zusätzlich den Abruf von Location Reports auf Lost Devices beschränken, die unter der gleichen Apple-ID registriert sind.
Eine Registrierung, welche die Zuordnung eines hochgeladenen Location Reports zu einer Apple-ID erlaubt, ist jedoch im Dienst aktuell nicht vorgesehen und würde eine signifikante Anpassung des Dienstes erfordern.
Entweder müsste der \ac{MBK} mit der Apple-ID verknüpft werden, was die Ende-zu-Ende-Verschlüsselung aufweicht, da Apple die Daten dann entschlüsseln könnte.
Alternativ müsste das Advertisement-Format um Daten zur Verknüpfung mit der Apple-ID erweitert werden.
Letzteres würde die Verfolgbarkeit von Lost Devices anhand der Advertisement-Pakete im Vergleich zum aktuellen System erleichtern.
Zusätzlich bietet bereits die aktuelle Implementierung keine Möglichkeit noch mehr Daten in einem einzelnen Advertisement-Paket zu senden.
Apple hat demnach keine Möglichkeit den Missbrauch einzuschränken, ohne signifikante Anpassungen des Dienstes vorzunehmen.
Da dieses Szenario die Privatsphäre der Nutzer nicht gefährdet wird auf die Erarbeitung möglicher Gegenmaßnahmen durch die Nutzer verzichtet.

\subsection{Gegenmaßnahmen zu \nameref{missbrauch:6}}
Um die Korrelation der Standortdaten zu verhindern, muss entweder der Upload oder der Download der Location Reports ohne Authentifizierung erfolgen.
Ein nicht authentifizierter Upload würde jedoch den Upload gefälschter von Location Reports erleichtern, da dafür kein Apple-Gerät mehr benötigt würde.
Die Auswirkungen gefälschter Reports sind mit den Auswirkungen von Replay-Angriffen nach Szenario \nameref{missbrauch:1} vergleichbar und könnten die Verfügbarkeit und Integrität der Standortdaten gefährden.
Der Download könnte prinzipiell ohne Authentifizierung erfolgen, da die Location Reports Ende-zu-Ende-verschlüsselt sind und auch die aktuelle Implementierung nicht prüfen kann, ob ein Location Report zur Apple-ID gehört, die den Download anfordert.
Diese Maßnahme wird auch von Heinrich \textit{et al.} \cite{Heinrich_FindMy} vorgeschlagen.
Es wäre jedoch möglich, dass Gegenmaßnahmen für andere Missbrauchsszenarien, wie zum Beispiel \nameref{missbrauch:3} und \nameref{missbrauch:4}, einen authentifizierten Download erforderlich machen.
Weiterhin ist fraglich, inwieweit Apple ein Interesse daran hat, die Korrelation der Standorte der Nutzer zu verhindern.
Es ist aktuell nicht bekannt, ob Apple dieses Missbrauchsszenario für eigene Zwecke nutzt.
Auch ist nicht klar, ob und wie häufig Strafverfolgungsbehörden die Daten zu „Wo ist?“ anfragen, da diese Informationen in Apples Transparenzbericht nicht gesondert aufgeführt werden \cite{Apple_Transparency}.

Es sind weiterhin keine Maßnahmen bekannt, wie Nutzer diesen Missbrauch verhindern oder erschweren können.
