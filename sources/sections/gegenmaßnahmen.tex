\section{Gegenmaßnahmen}
\label{sec:Gegenmassnahmen}

Im Folgenden werden mögliche Gegenmaßnahmen gegen die Szenarien \nameref{missbrauch:1} bis \nameref{missbrauch:6} vorgestellt.


\subsection{Gegenmaßnahmen zu \nameref{missbrauch:1}}
Das Design des Dienstes lässt es prinzipiell zu, Replay-Angriffe zumindest erkennen zu können.
Die „Wo ist?“ App könnte beispielsweise die Plausibilität der empfangenen Standortdaten prüfen und den Nutzer auf eventuelle Manipulationen hinweisen.
Jedoch kann die App nicht in jedem Fall erkennen, welche Location Reports manipuliert und welche gültig sind.
Eine solche Prüfung wäre unter anderem anhand der Zeitpunkte der Standortdaten möglich.
Große Abweichungen des Standorts bei Reports, welche kurz nacheinander oder gleichzeitig erstellt wurden, könnten als nicht realistisch erkannt werden.
Allerdings lassen sich so nicht alle Manipulationen erkennen, da der Replay-Angriff zum Beispiel auch in der Nähe des tatsächlichen Standorts durchgeführt werden kann, sodass die Abweichungen der Standorte klein sind.
Dennoch wäre die Position des Lost Device in diesem Fall nicht genau bestimmbar.
Außerdem könnte das Owner Device bestimmen, welcher Advertising Key zum Zeitpunkt der Generierung des Location Reports aktiv war und so überprüfen ob der Location Report mit dem passenden Advertising Key erstellt wurde, oder ob ein älterer Schlüssel verwendet wurde.
So lassen sich mögliche Replay-Angriffe auf das Intervall der Schlüsselrotation beschränken.

Jedoch sind diese Maßnahmen nur durch Apple, durch eine Aktualisierung der „Wo ist?“ App umsetzbar. 
Auf Apples Server können Manipulationen hingegen nicht erkannt werden, da dazu die Standortdaten entschlüsselt werden müssten.

Die Nutzer haben keine Möglichkeit, sich vor einem Replay-Angriff zu schützen.
Deaktivieren des „Wo ist?“ Dienstes verhindert zwar, dass manipulierte Standortdaten empfangen werden, jedoch können gar keine Standortdaten mehr empfangen werden.

\subsubsection{Gegenmaßnahmen zu \nameref{missbrauch:2.1} und \nameref{missbrauch:2.2}}

Gegen physische Angriffe auf ein Lost Device kann das Design des „Wo ist?“ Dienstes nicht schützen.
Demnach hat Apple kaum Möglichkeiten, die Nutzer vor solchen physischen Angriffen zu schützen.
Die Zerstörung des Geräts könnte maximal erschwert, aber nicht verhindert werden.
Durch eine fest verbaute Batterie könnte das Entfernen der Batterie ebenfalls nur erschwert aber nicht verhindert werden.

Indirekt lassen sich physische Angriffe weiter erschweren, indem die Advertisement-Intervalle verlängert werden, wodurch ein Angreifer mehr Zeit benötigen würde, um das Lost Device zu finden.
Diese Maßnahme würde allerdings gleichzeitig das Tracking anderer Personen unter Verwendung eines versteckten Trackers (\nameref{missbrauch:3}) erleichtern.
Generell ist der „Wo ist?“ Dienst nicht als Hilfsmittel gegen Diebstahl konzipiert und wird auch nicht als solches beworben \cite{Apple_WoIst}, weshalb Apple hier keine weiteren Maßnahmen ergreift.

Schutz gegen die Entladung der Batterie durch Angreifer in der Nähe ist ebenfalls nur schwierig umsetzbar.
Da bei Apples System die Verschlüsselung nicht auf dem verlorenen Gerät, sondern auf dem Finder Device erfolgt, ist dieser Angriff bereits aufwändiger als beim System von Garg \textit{et al.} \cite{Garg_Secure_Tracker}.
Außerdem ist das Abspielen eines Tons vergleichsweise auffällig und damit generell für Angreifer eher nicht geeignet.
Zusätzlich könnte auf die Funktion einen Ton abzuspielen komplett verzichtet werden, was den Angriff weiter erschweren würde.
Allerdings ist diese Funktion Teil der Schutzmaßnahmen gegen direktes Tracking und die Entfernung dieser Funktion würde die Nutzer schlechter vor direktem Tracking schützen.

Die Nutzer haben ebenfalls kaum Möglichkeiten sich gegen die Szenarien \nameref{missbrauch:2.1} und \nameref{missbrauch:2.2} zu schützen. 
Das Entfernen des Lautsprechers von Accessories wäre eine Möglichkeit die Entladung zu erschweren.
Da die Entladung durch das Abspielen von Tönen jedoch generell als eher unwahrscheinlich anzusehen ist, ist diese Maßnahme vermutlich nicht notwendig.


\subsubsection{Gegenmaßnahmen zu \nameref{missbrauch:3}}
% TODO: genauere Beschreibung des Anti-Tracking systems, eventuell mehr von missbrauch hierher

Inzwischen bietet Apple eine App für Android an, um die Warnung vor Tracking auch für Android-Nutzer zu ermöglichen, welche jedoch von Heinrich \textit{et al.} \cite{Heinrich_AirGuard} mangels Features nicht als ausreichend angesehen wird.
Die Drittanbieter-App \textit{AirGuard} für Android bietet eine ähnliche Warnfunktion, welche in Tests besser abschneidet als die Warnfunktion von iOS \cite{Heinrich_AirGuard}.
Insbesondere erkennt nur AirGuard das Tracking durch inoffizielle Tracker.
Solche Tracker, die das „Wo ist?“ Netzwerk ohne explizite Erlaubnis von Apple nutzen, können von Angreifern leicht aus kostengünstiger \ac{BLE}-Hardware und Open-Source Software selbst gebaut werden.
Diese können die von Mayberry \textit{et al.} \cite{Mayberry_Tracking} vorgestellten Maßnahmen nutzen, um die offizielle Unwanted tracking detection zu umgehen.
Inoffizielle Tracker, welche ein Intervall von weniger als einer Stunde für die Schlüsselrotation verwenden, können allerdings weder von der offiziellen Warnfunktion noch von AirGuard erkannt werden \cite{Heinrich_AirGuard}.
