\section{Missbrauch des „Wo ist?“ Dienstes}
\label{sec:Missbrauch}

Das Hauptangriffsziel für Angreifer des „Wo ist?“ Dienstes sind die Standortdaten der Nutzer.
Apple ergreift verschiedene Maßnahmen, um die Sicherheit der Standortdaten sicherzustellen, wie bei der Erläuterung der Funktionsweise in \autoref{sec:Funktionsweise_FindMy} bereits gezeigt wurde.
Dennoch besteht Missbrauchspotenzial, gegen welches der Dienst nicht oder nicht ausreichend geschützt ist, wie unter anderem durch Tonetto \textit{et al.} \cite{Tonetto_FindMy} aufgezeigt.

Bevor konkrete Missbrauchsszenarien präsentiert werden, gegen welche Apple unzureichende Maßnahmen ergreift, sollen zunächst die grundlegenden Sicherheitsziele \textit{Confidentiality}, \textit{Integrity} und \textit{Availability} und mögliche Angriffe auf diese Ziele erklärt werden.




% zu schützendes Asset: Standortdaten
% Sicherheitsziele: CIA
%   Confidentiality: Abruf der Standortdaten durch Angreifer (direktes Abrufen von Standortdaten eines Nutzers von Apples server praktisch nicht möglich durch Verschlüsselung, aber über Umwege direktes und indirektes Tracking möglich)
%   Integrity: Manipulation der Standortdaten durch Angreifer (Replay Angriff mit Advertising Keys möglich, manipuliert die gespeicherten Standortdaten, schon auf Server gespeicherte Daten können nur sehr schwer manipuliert werden)
%   Availability: Abrufen der Standortdaten verhindern (Angriff auf Apples Server ist eher unrealistisch, Akku der AirTags entleeren im Vergleich zu anderen Systemen schwer, da AirTags allgemein  nur advertisen)


% Konkrete Szenarien:
% - Tracking von Personen
%   - Direkt: Tracking durch AirTags/sonstige kleine Geräte ("Tracker")
%   - Indirekt: wie von Tonetto et al. gezeigt -> über Tracker in der Umgebung, weniger genau
% - Verdeckter Datentransfer
%   - wie von Tonetto et al. und bei SendMy gezeigt
% - Korrelation der Positionen durch Apple, eher theoretisch
