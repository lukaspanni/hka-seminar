\section{Grundlagen}
\label{sec:grundlagen}

Die folgenden Abschnitte dienen als kurze Einführung zu den technologischen Grundlagen von Apples „Wo ist“ Dienst.
Dazu werden insbesondere der Standard zu kabellosen Kurzstreckenkommunikation \ac{BLE} und die asymmetrische Verschlüsselung auf Basis elliptischer Kurven, auch \ac{ECC}, vorgestellt.

\subsection{\ac{BLE}}
\label{sec:ble}
Für die Erklärung der Funktionsweise ist insbesondere das \ac{BLE}-Advertising relevant.
Der Standard für \ac{BLE} ist ein Teil des Bluetooth Standards und auf Kurzstreckenkommunikation bei geringem Energieverbrauch ausgelegt.
Das Advertising wird dazu verwendet, das Gerät für andere Geräte in der Umgebung auffindbar zu machen.
Dazu werden Advertisement-Pakete im Abstand von einem sogenannten \textit{Advertising-Intervall} gesendet \cite{Spec_BLE_5.3}.
Für das Advertising stellt der \ac{BLE}-Standard drei der 40 verfügbaren Kanäle im Frequenzbereich von 2,4 GHz bis 2,4835 GHz zur Verfügung.
Jedes Advertisement-Paket wird nacheinander, mit einer zufälligen Verzögerung zur Kollisionsvermeidung, in jedem der drei Kanäle gesendet.
Das minimale Advertising-Intervall beträgt 20 Millisekunden, das maximale 10,24 Sekunden \cite{Spec_BLE_5.3}.
Je länger das Intervall, desto geringer ist der Energieverbrauch des Geräts, das die Advertisement-Pakete sendet.
Allerdings leidet die Auffindbarkeit unter langen Intervallen.
Apple verwendet für den „Wo ist“ Dienst als Kompromiss zwischen optimaler Auffindbarkeit und minimalem Energieverbrauch ein Advertising-Intervall von zwei Sekunden \cite{Apple_FindMySpec}.


Der \ac{BLE}-Standard erlaubt es, beliebige Daten in einem Advertisement-Paket zu versenden, sofern der Typ des Pakets als \textit{manufacturer-specific-data} festgelegt wird.
In diesem Fall kann das Paket 27 Byte Nutzdaten enthalten \cite{Spec_BLE_5.3}.
Da Apple für andere Dienste, wie beispielsweise AirDrop, ebenfalls \ac{BLE} nutzt, kommt ein proprietäres Format für die Nutzdaten zum Einsatz \cite{Heinrich_FindMy}.
Das genaue Format für den „Wo ist“ Dienst wird in \autoref{fig:apple_advertising} vorgestellt.

Um Advertisement-Pakete zu empfangen, müssen die Advertising-Kanäle abgehört werden.
Dazu wechseln die Geräte jeweils auf einen der Kanäle und erfassen Advertisement-Pakete für eine definierte Zeit.
Anschließend wird nach einer Wartezeit auf den nächsten Kanal gewechselt \cite{Spec_BLE_5.3}.
Wie häufig und wie lange Apple-Geräte im Hintergrund die Kanäle nach Advertisement-Paketen anderer Apple-Geräte abhören, ist nicht bekannt.

\section{\ac{ECC}}
\label{sec:ecc}
Auf elliptischen Kurven basierende Kryptografieverfahren gehören zu den asymmetrischen Verschlüsselungsverfahren.
Asymmetrisch sind alle Verfahren, welche für die Verschlüsselung und die Entschlüsselung zwei verschiedene Schlüssel verwenden.
Der für die Verschlüsselung verwendete Schlüssel muss dabei nicht geheim gehalten werden, da verschlüsselte Nachrichten mit diesem nicht entschlüsselt werden können.
Der Schlüssel kann öffentlich zugänglich gemacht werden.
Daher heißt der Verschlüsselungsschlüssel auch öffentlicher Schlüssel oder \textit{public key}.
Der Entschlüsselungsschlüssel, auch privater Schlüssel oder \textit{private key} genannt, muss hingegen geheim gehalten werden \cite[S. 173ff.]{Krypto}.

Die asymmetrische Verschlüsselung löst somit das Problem des Schlüsselaustauschs bei symmetrischen Verschlüsselungsverfahren, bei denen nur ein Schlüssel existiert, welcher geheim gehalten werden muss.
Jedoch sind asymmetrische Verfahren meist rechenintensiver und damit langsamer als symmetrische Verfahren mit vergleichbarem Sicherheitsniveau.
Oft werden deshalb Hybride Verfahren verwendet \cite[S. 178f.]{Krypto}.
Auch für die Verschlüsselung der Standortdaten in Apples „Wo ist“ Dienst wird ein hybrides Verfahren verwendet, welches in \autoref{sec:Funktionsweise_FindMy} näher erläutert wird.

Für den „Wo ist“ Dienst wird \ac{ECC} verwendet.
Verfahren auf Basis von elliptischen Kurven erreichen ein vergleichbares Sicherheitsniveau mit kürzeren Schlüsseln als andere asymmetrische Verfahren \cite[S. 273.]{Krypto}.
Dieser Vorteil wird ausgenutzt, damit der öffentliche Schlüssel in \ac{BLE} Advertisement-Paketen übertragen werden kann.
Meist kommen standardisierte elliptische Kurven zum Einsatz, Apple nutzt beispielsweise die NIST P-224 Kurve \cite{Heinrich_FindMy}.
Jede Kurve wird durch eine Primzahl, zwei Kurvenkoeffizenten und einen primitiven Punkt auf der Kurve definiert.
Der private Schlüssel ist eine zufällige ganze Zahl.
Durch die Multiplikation des privaten Schlüssels mit dem primitiven Punkt wird der öffentliche Schlüssel als Punkt auf der Kurve berechnet.
Der öffentliche Schlüssel wird in Form der X und Y Koordinate des Punktes angegeben.
Weiterhin lässt sich der öffentliche Schlüssel in der Praxis auf eine Koordinate reduzieren \cite[S. 284f.]{Krypto}.

Schlüssel werden oft nur für einen begrenzten Zeitraum verwendet, wodurch sich die Sicherheit erhöhen kann \cite[S. 378f.]{Krypto}.
Wenn der Schlüssel für die Verschlüsselung der Standortdaten regelmäßig gewechselt wird, kann ein Angreifer, auch wenn er einen Schlüssel erlangt, nur die Standortdaten aus einer begrenzten Zeit abrufen.
Weiterhin kann die Zuordnung von öffentlichem Schlüssel zu einem Gerät durch häufiges Wechseln der Schlüssel erschwert werden.
\ac{KDF} können genutzt werden, um aus bestehenden Schlüsseln und einer Zusatzinformation neue Schlüssel zu erzeugen.
Dabei kann jeder, der den Schlüssel und die Zusatzinformation kennt, neue Schlüssel ableiten \cite[S. 378ff.]{Krypto}.
Im „Wo ist“ Dienst wird eine \ac{KDF} genutzt, um aus einem geheimen Schlüssel und einem Zähler regelmäßig neue Schlüsselpaare abzuleiten \cite{Heinrich_FindMy}.
