\section{Funktionsweise des „Wo ist?“ Dienstes}
\label{sec:Funktionsweise}

Mit der Einführung ihres ersten \ac{BLE}-Trackers hat die Firma Tile 2013 das Konzept eines \ac{BLE}-basierten Tracking-Systems populär gemacht.
Wie viele konkurrierende Systeme, darunter Apples „Wo ist?“ Dienst, setzt auch Tile auf crowdsourcing zur Bestimmung der Position verlorener Gegenstände \cite{Weller_BLE_Finders}.
Die grundlegende Funktion dieser Systeme ist jeweils identisch: Ein zu findendes Gerät sendet periodisch \ac{BLE}-Advertisements, die von Smartphones, welche am entsprechenden System teilnehmen, empfangen werden können.
Die Empfänger können ihre Position über das \ac{GPS} bestimmen und diese zusammen mit einer ID, welche das zu findende Gerät identifiziert, an einen Server übermitteln.
Der Besitzer des verlorenen Geräts kann die Position seines Geräts vom Server abrufen \cite{Garg_Secure_Tracker}.



Apples „Wo ist?“ Dienst 