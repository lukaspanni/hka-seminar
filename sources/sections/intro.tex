
\section{Einleitung}

Apples „Wo ist?“ Dienst erlaubt es seinen Nutzern, ihre Apple-Geräte zu lokalisieren.
Der Dienst ist auch in der Lage die Position von Geräten zu bestimmen, welche nicht mit dem Internet verbunden sind.
Dazu nutzt Apple die große Verbreitung seiner Geräte, welche den Standort von anderen Apple-Geräten an Apple übermitteln können.
Bei 62,6 Millionen Smartphone Nutzern in Deutschland \cite{Statista_SmartphonesDeutschland} und einem iPhone Anteil von 35,3 \% \cite{Statscounter_Marktanteil_iOS} gibt es in Deutschland knapp 22 Millionen iPhones, welche standardmäßig die Standortdaten anderer an Apple senden.
Zu den iPhones kommen allerdings noch iPads und MacBooks, die ebenfalls standardmäßig das „Wo ist?“ Netzwerk unterstützen.

Obwohl die Standortdaten pseudonymisiert und verschlüsselt werden, lässt sich das „Wo ist?“ Netzwerk für Zwecke missbrauchen, welche nicht von Apple vorgesehen sind.
Dazu zählen unter anderem die Verfolgung von Personen, die Ausnutzung für eine versteckte Datenübertragung und die Verwendung als Tool zur Analyse von Menschenströmen.
Insbesondere die Verfolgung von Personen ist ein starker Eingriff in die Privatsphäre der betroffenen Nutzer.
Durch die große Verbreitung von Apple-Produkten sind viele Menschen potenzielle Opfer eines solchen Missbrauchs des „Wo ist?“ Netzwerks.
Darüber hinaus beschränken sich einzelne mögliche Angriffe nicht nur auf die Nutzer von Apple-Geräten, sondern können auch die Privatsphäre anderer Personen beeinträchtigen.

% TODO: Weiter ausführen sobald andere Abschnitte weiter sind
In den folgenden Abschnitten werden zunächst relevante Grundlagen zur Privatsphäre im \ac{IOT} allgemein, zu \ac{BLE} und zu crowdsourced-tracking basierend auf \ac{BLE} eingeführt. 
In \autoref{sec:Funktionsweise_FindMy} wird auf Grundlage der allgemeinen Funktionsweise von crowdsource-tracking, das Funktionsprinzip des „Wo ist?“ Dienstes genauer vorgestellt.
Im Anschluss wird in \autoref{sec:Missbrauch} aufgezeigt, wie das „Wo ist?“ Netzwerk für verschiedene, nicht von Apple vorgesehene, Anwendungszwecke missbraucht werden kann.
Abschließend sollen in \autoref{sec:Gegenmaßnahmen} mögliche Gegenmaßnahmen diskutiert werden.
Dabei soll gezeigt werden, inwiefern Nutzer des Systems sich selbst schützen können, und an welchen Stellen Apple das System anpassen muss, um Schwachstellen zu beseitigen.
%TODO: auf Sicherheits-Properties definiert von Garg et al. und als sicher vorgestellte Systeme (Weller et al., Garg et al. -> eventuell diese miteinander vergleichen) eingehen