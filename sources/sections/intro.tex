
\section{Einleitung}

Apples „Wo ist?“ Dienst wurde dazu entwickelt, verlorene Apple-Geräte zu lokalisieren.
Der Dienst nutzt dazu das Funktionsprinzip des Crowdsourced-Tracking und kann damit auch Geräte ohne Internetverbindung finden.
Wie andere Dienste mit ähnlicher Funktionalität verwendet Apple \ac{BLE}.
Verlorene Geräte senden periodisch Daten über \ac{BLE} an alle Geräte in der Nähe.
Durch die große Verbreitung seiner Geräte kann Apple davon ausgehen, dass in den meisten Fällen andere Apple-Geräte in der Nähe sind, welche die Daten empfangen und den Standort des verlorenen Gerätes bestimmen können.
Bei 62,6 Millionen Smartphone Nutzern in Deutschland \cite{Statista_SmartphonesDeutschland} und Apples Marktanteil von 35,3 \% \cite{Statscounter_Marktanteil_iOS} gibt es alleine in Deutschland knapp 22 Millionen iPhones.
Standardmäßig ist der Dienst bei allen iPhones aktiviert, sodass diese das Finden verlorener Geräte aktiv unterstützen.
Weiterhin ist der Dienst bei allen iPads und MacBooks ebenfalls im Auslieferungszustand aktiviert.


Obwohl Apple verschiedene Sicherheitsmaßnahmen ergreift und die Standortdaten beispielsweise pseudonymisiert und verschlüsselt übertragen werden, lässt sich der „Wo ist?“ Dienst für Zwecke missbrauchen, welche nicht von Apple vorgesehen sind.
Dazu zählen unter anderem die Verfolgung von Personen, die Ausnutzung für die kostenfreie versteckte Übertragung von beliebigen Daten und die Verwendung zur Analyse von Menschenströmen.
Insbesondere die Verfolgung stellt einen starken Eingriff in die Privatsphäre der betroffenen Personen dar.
Durch die große Verbreitung von Apple-Produkten wird der Dienst von vielen Menschen genutzt, was den Missbrauch teilweise einfacher und für Angreifer attraktiver macht.
Die Auswirkungen der hier vorgestellten Missbrauchsszenarien beschränken sich darüber hinaus nicht nur auf die Nutzer von Apple-Geräten, sondern können teilweise für alle potenziell gefährlich sein.
Für Personen, die sich schützen wollen, existieren nur wenige Gegenmaßnahmen, welche zusätzlich nicht jeden Missbrauch abdecken und teilweise leicht umgangen werden können.
Stattdessen müsste Apple an vielen Stellen nachbessern, um die Privatsphäre aller ausreichend zu schützen.


Zunächst werden in \autoref{sec:grundlagen} einige relevante Grundlagen zu den Basistechnologien des „Wo ist?“ Dienstes, \ac{BLE} und \ac{ECC} erläutert.
Darauf aufbauend wird in \autoref{sec:Funktionsweise} die allgemeine Funktionsweise von Crowdsourced-Tracking mittels \ac{BLE} erläutert, die sowohl Apples Implementierung als auch Konkurrenzdiensten zugrunde liegt.
In \autoref{sec:Funktionsweise_FindMy} werden das Funktionsprinzip des „Wo ist?“ Dienstes und die Unterschiede zu anderen Diensten genauer vorgestellt.
Dabei liegt ein besonderer Fokus auf der Umsetzung von Sicherheitsmaßnahmen, welche die Standortinformationen der Nutzer vor unbefugtem Zugriff schützen sollen.
Im Anschluss werden in \autoref{sec:Missbrauch} allgemeine Ziele beim Angriff auf den Dienst, sowie die Auswirkungen der vorhandenen Gegenmaßnahmen erläutert.
Darauf aufbauend werden in \autoref{sec:szenarien} sechs konkrete Szenarien und mögliche Gegenmaßnahmen vorgestellt.
Dabei soll gezeigt werden, inwiefern Nutzer des Systems sich selbst schützen können, und an welchen Stellen Apple das System anpassen muss, um vorhandene Schwachstellen zu beseitigen, falls dies möglich ist.
Abschließend werden in \autoref{sec:Fazit} die Inhalte dieser Arbeit zusammengefasst und mögliche offene Fragen aufgezeigt. 