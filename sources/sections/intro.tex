
\section{Einleitung}

Apples „Wo ist?“ Dienst wurde dazu entwickelt, verlorene und vergessene Apple-Geräte zu lokalisieren.
Dabei ist der Dienst in der Lage nicht nur Geräte mit Internetverbindung zu lokalisieren, sondern auch Geräte zu finden, die aktuell keine Verbindung haben oder nicht internetfähig sind.

Zur Positionsbestimmung von Geräten ohne Internetverbindung nutzt Apple die große Verbreitung seiner Geräte aus.
Bei 62,6 Millionen Smartphone Nutzern in Deutschland \cite{Statista_SmartphonesDeutschland} und Apples Marktanteil von 35,3 \% \cite{Statscounter_Marktanteil_iOS} gibt es in Deutschland knapp 22 Millionen iPhones, welche den „Wo ist?“ Dienst standardmäßig unterstützen.
Zudem nehmen auch iPads und MacBooks ohne anderslautende Einstellung am Netzwerk des Dienstes teil und unterstützen das Finden verlorener Geräte.

iPhones, iPads und MacBooks können Geräte in der Nähe erkennen, welche den „Wo ist?“ Dienst nutzen und den aktuellen Standort speichern und an Apple übermitteln.
Wie andere Dienste mit ähnlicher Funktionalität verwendet Apple zur Erkennung von Geräten in der Nähe den \ac{BLE}-Standard.


Obwohl die Standortdaten pseudonymisiert und verschlüsselt werden, lässt sich der „Wo ist?“ Dienst für Zwecke missbrauchen, welche nicht von Apple vorgesehen sind.
Dazu zählen unter anderem die Verfolgung von Personen, die Ausnutzung für die kostenfreie versteckte Übertragung von beliebigen Daten und die Verwendung zur Analyse von Menschenströmen.
Insbesondere die Verfolgung stellt einen starken Eingriff in die Privatsphäre der betroffenen Personen dar.
Durch die große Verbreitung von Apple-Produkten sind viele Menschen potenzielle Opfer eines solchen Missbrauchs des „Wo ist?“ Netzwerks.
Darüber hinaus beschränkt sich eine Möglichkeit der Verfolgung nicht auf Nutzer des Dienstes, sondern kann zur Verfolgung von beliebigen Personen und Gegenständen missbraucht werden.
Für Personen, die sich schützen wollen, existieren nur wenige Gegenmaßnahmen, welche zusätzlich nicht jeden Missbrauch abdecken und teilweise leicht umgangen werden können.


In den folgenden Abschnitten werden zunächst einige relevante Grundlagen zu den Basistechnologien des „Wo ist?“ Dienstes, \ac{BLE} und \ac{ECC} erläutert.
Darauf aufbauend wird die allgemeine Funktionsweise von crowdsourced-tracking mittels \ac{BLE} erläutert, die sowohl Apples Implementierung als auch Konkurrenzdiensten zugrunde liegt.
In \autoref{sec:Funktionsweise_FindMy} werden das Funktionsprinzip des „Wo ist?“ Dienstes und die Unterschiede zu anderen Diensten genauer vorgestellt.
Dabei liegt ein besonderer Fokus auf der Umsetzung von Sicherheitsmaßnahmen, welche die Standortinformationen der Nutzer vor unbefugtem Zugriff schützen sollen.
Im Anschluss wird in \autoref{sec:Missbrauch} an ... konkreten Szenarien aufgezeigt, wie der „Wo ist?“ Dienst, trotz Gegenmaßnahmen, von Dritten und von Apple selbst missbraucht werden kann.
Abschließend werden in \autoref{sec:Gegenmassnahmen} mögliche Gegenmaßnahmen für jedes Szenario erarbeitet und diskutiert.
Dabei soll gezeigt werden, inwiefern Nutzer des Systems sich selbst schützen können, und an welchen Stellen Apple das System anpassen muss, um vorhandene Schwachstellen zu beseitigen, falls dies möglich ist.